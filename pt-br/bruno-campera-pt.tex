\documentclass[a4paper,12pt]{article}

%A Few Useful Packages
\usepackage{marvosym}
\usepackage{fontspec} 					%for loading fonts
\usepackage{xunicode,xltxtra,url,parskip} 	%other packages for formatting
\RequirePackage{color,graphicx}
\usepackage[usenames,dvipsnames]{xcolor}
%\usepackage[big]{layaureo} 				%better formatting of the A4 page
\usepackage{fullpage}			%better formatting of the A4 page
% an alternative to Layaureo can be ** \usepackage{fullpage} **
\usepackage[a4paper, total={7.4in, 10in}]{geometry}
\usepackage{supertabular} 				%for Grades
\usepackage{titlesec}					%custom \section
\usepackage{fontawesome}                %icons

\usepackage{multirow}

\usepackage{paralist} % for horizontal bullet points
\usepackage[inline]{enumitem} % load package
\usepackage{longtable} % for the work experience section allowing page breaks

%Setup hyperref package, and colours for links
\usepackage{hyperref}
\newcommand{\preheadingspacing}{\vspace{6pt}}
\newcommand{\postheadingspacing}{\vspace{6pt}}
\definecolor{linkcolour}{rgb}{0,0.2,0.7}
\hypersetup{colorlinks,breaklinks,urlcolor=linkcolour, linkcolor=linkcolour}

%FONTS
\defaultfontfeatures{Mapping=tex-text}
%\setmainfont[SmallCapsFont = Fontin SmallCaps]{Fontin}
%%% modified for Karol Kozioł for ShareLaTeX use
\setmainfont[
SmallCapsFont = Fontin-SmallCaps.ttf,
BoldFont = Fontin-Bold.ttf,
ItalicFont = Fontin-Italic.ttf
]
{Fontin-Regular.ttf}
%%%

%CV Sections inspired by:
%http://stefano.italians.nl/archives/26
\titleformat{\section}{\Large\scshape\raggedright}{}{0em}{}[\titlerule]
\titlespacing{\section}{0pt}{3pt}{3pt}
%Tweak a bit the top margin
\addtolength{\voffset}{.6cm}

%Italian hyphenation for the word: ''corporations''
\hyphenation{im-pre-se}


%-------------WATERMARK TEST [**not part of a CV**]---------------
\usepackage[absolute]{textpos}

\setlength{\TPHorizModule}{30mm}
\setlength{\TPVertModule}{\TPHorizModule}
\textblockorigin{2mm}{0.65\paperheight}
\setlength{\parindent}{0pt}

\setlist[itemize]{itemsep=0pt, topsep=0pt}

%--------------------BEGIN DOCUMENT----------------------
\begin{document}

%WATERMARK TEST [**not part of a CV**]---------------
%\font\wm=''Baskerville:color=787878'' at 8pt
%\font\wmweb=''Baskerville:color=FF1493'' at 8pt
%{\wm
%	\begin{textblock}{1}(0,0)
%		\rotatebox{-90}{\parbox{500mm}{
%			Typeset by Alessandro Plasmati with \XeTeX\  \today\ for
%			{\wmweb \href{http://www.aleplasmati.comuv.com}{aleplasmati.comuv.com}}
%		}
%	}
%	\end{textblock}
%}

\pagestyle{empty} % non-numbered pages

\font\fb=''[cmr10]'' %for use with \LaTeX command

%--------------------TITLE-------------
\par{
    \centering{\LARGE Bruno \textsc{Campera}}\\
    \textcolor{white}{\tiny\textbf{Contato}}\\
\par}

\noindent\makebox[\textwidth]{
\begin{tabular}{rlrl}
\textsc{Email:}     & brunoc@alunos.utfpr.edu.br
    & \textsc{Links:} & \href{https://www.linkedin.com/in/bruno-campera-288b0b77/}{\faLinkedinSquare\ LinkedIn}  \href{https://github.com/brunocmpr}{\faGithubSquare\ GitHub} \\
\textsc{Localização:} & Curitiba/PR, Brazil.
    & \textsc{Celular:}     & \href{https://api.whatsapp.com/send?phone=5541996658096}{\faWhatsapp\ +55~41~99665~8096}\\
\end{tabular}
}
\par{
    \textcolor{white}{\tiny\textbf{Resumo}}\\
    \centering{\normalsize Engenheiro de Software Backend especializado em Java, Spring Boot, Quarkus, AWS, com paixão por construir sistemas escaláveis e de fácil manutenção. Há mais de 7 anos de experiência na área de tecnologia, incluindo colaboração com equipes internacionais para entrega de soluções robustas.}
\par}


% \section{About me}
% Computer Engineer graduated from the Federal Technological University of Paraná (UTFPR).
% I am a backend software engineer, using Java and Spring Boot or Quarkus, using tools like Hibernate, Spring Data JPA, JUnit and Mockito.
% I intend to deepen my skills in high-level web programming technologies and develop expertise in infrastructure to ensure end-to-end delivery, from  gathering solution requirements to performing a working deployment.

%Computer Engineer graduated from the Federal Technological University of Paraná (UTFPR). Recently, I worked as a web developer, using Java and Spring Boot on the back-end and Angular on the front-end. Formerly, I was a developer of embedded systems. Now, I intend to deepen my skills in high-level web programming technologies and develop expertise in infrastructure to ensure end-to-end delivery, from solution requirements to functional deployment.
% Computer Engineer from the Federal University of Technology - Paraná (UTFPR). I have recently worked as a web developer, using Java and Spring Boot for the back-end and Angular for the front-end. Previously, I worked as an embedded systems developer, I am now eager to broaden my skill set by delving into higher-level web programming technology stacks and further developing expertise in infrastructure.
%Engenheiro de Computação graduado pela Universidade Tecnológica Federal do Paraná (UTFPR) com paixão pela área de desenvolvimento de software. Possui interesse na integração software-hardware de sistemas embarcados e na conectividade da qual usufruem sistemas web e aplicações distribuídas.

%Engenheiro de Computação graduado pela Universidade Tecnológica Federal do Paraná (UTFPR) com paixão pela área de desenvolvimento. Possui interesse na integração software-hardware, especialmente através de sistemas embarcados. Possui experiência profissional em desenvolvimento em linguagens de alto nível para plataforma ERP SAP e, atualmente, busca transição de sua atuação profissional para o desenvolvimento de sistemas embarcados, com programação em linguagens de baixo nível.

% compreendendo, se possível, a atuação como analista funcional.

%Estudante de Engenharia de Computação na Universidade Tecnológica Federal do Paraná, natural de São Paulo, SP. Possui facilidade em trabalho em equipe. Interesse em desenvolvimento de sistemas embarcados em microcontroladores, com conhecimentos prévios com plataforma ARM Cortex-M3 e Arduíno.

%Section: Work Experience at the top
\preheadingspacing
\section{Habilidades técnicas}
\postheadingspacing
\begin{tabular}{rp{15.2cm}}
\raggedleft Programação:
& \begin{itemize*}[label=\Large\textbullet]
    \item Java
    \item Javascript
    \item Dart
    \item Desenvolvimento de APIs RESTful
\end{itemize*}\\
\raggedleft Frameworks:
& \begin{itemize*}[label=\Large\textbullet]
    \item Spring Boot
    \item Quarkus
    \item Flutter
\end{itemize*}\\
\raggedleft Cloud:
& \begin{itemize*}[label=\Large\textbullet]
    \item AWS
\end{itemize*}\\
\raggedleft Banco de dados e ORM:
& \begin{itemize*}[label=\Large\textbullet]
    \item PostgreSQL
    \item MySQL
    \item Hibernate
    \item Spring Data JPA
    \item PanacheORM
\end{itemize*}\\
\raggedleft Ferramentas adicionais:
& \begin{itemize*}[label=\Large\textbullet]
    \item JUnit 5
    \item Mockito
    \item Linux
    \item Git
\end{itemize*}\\
\raggedleft Processos:
& \begin{itemize*}[label=\Large\textbullet]
    \item Metodologias ágeis, com destaque para vivência em Scrum e Kanban
\end{itemize*}\\
\raggedleft Línguas:
& \begin{itemize*}[label=\Large\textbullet]
    \item Inglês (C1 Avançado)
    % \item Português brasileiro nativo %Not necessary for 
\end{itemize*}\\
\end{tabular}

\preheadingspacing

\section{Experiência Profissional}
\vspace{8pt}

\textbf{\textsc{Engenheiro de Software}} -- \textbf{Stoneridge} \hfill \textit{07/2024 -- Atual}

{\small
\begin{itemize}[leftmargin=*,label=\large\textbullet]
    \setlength\itemsep{-0.2em}
    \item Desenvolvi e mantive APIs REST no backend com Java 17 e Spring, hospedadas em AWS EC2.
    \item Criei AWS Lambdas para aplicações serverless com Java 21 e Quarkus.
    \item Desenvolvi testes unitários utilizando JUnit 5 e Mockito.
    % \item Interagi com AWS Kinesis para intercomunicação de dispositivos e AWS DynamoDB para armazenamento de mensagens.
    \item Modelei e implementei um esquema de banco de dados MySQL utilizando SQL.
    \item Empreguei o uso de Hibernate para modelagem de entidades e gerenciamento de persistência.
    \item Utilizei Git para controle de versão.
\end{itemize}
}

\textbf{\textsc{Engenheiro de Software}} -- \textbf{FS Tech} \hfill \textit{12/2023 -- 12/2024}

{\small
\begin{itemize}[leftmargin=*,label=\large\textbullet]
    \setlength\itemsep{-0.2em}
    \item Desenvolvi APIs RESTful no backend com Java 17 e Quarkus.
    \item Modelei e implementei um esquema de banco de dados PostgreSQL utilizando SQL.
    \item Empreguei o uso de Hibernate ORM e Panache para interação com o banco de dados, juntamente com JPQL e SQL Nativo.
    \item Integrei Hibernate Envers para auditoria e versionamento de entidades do banco de dados.
    \item Utilizei Git para controle de versão.
\end{itemize}
}

\textbf{\textsc{Engenheiro de Software}} -- \textbf{Serviço Federal de Processamento de Dados (SERPRO)} \hfill \textit{09/2021 -- 09/2023}

{\small
\begin{itemize}[leftmargin=*,label=\large\textbullet]
    \setlength\itemsep{-0.2em}
    \item Trabalhei no desenvolvimento da plataforma federal de compras públicas Compras.gov.br.
    \item Desenvolvi APIs RESTful no backend utilizando Java 17 e Spring Boot 3.
    \item Modelei e implementei um esquema de banco de dados PostgreSQL usando SQL.
    \item Empreguei o uso de Hibernate ORM e Spring Data JPA para interação com o banco de dados, juntamente com JPQL.
    \item Contribuí para o desenvolvimento de uma SPA usando Angular, Typescript, HTML e CSS.
    \item Participei de time ágil seguindo Scrum.
    \item Utilizei Git para controle de versão.
\end{itemize}
}

\textbf{\textsc{Engenheiro de Software Embarcado}} -- \textbf{Alta Rail Technology} \hfill \textit{02/2021 -- 08/2021}

{\small
\begin{itemize}[leftmargin=*,label=\large\textbullet]
    \setlength\itemsep{-0.2em}
    \item Trabalhei no desenvolvimento de uma aplicação embarcada para trens ferroviários.
    \item Desenvolvi módulos em C++ para uma aplicação distribuída.
    \item Utilizei a biblioteca FlatBuffers do Google para serialização e intercomunicação entre módulos.
    \item Utilizei Git para controle de versão.
\end{itemize}
}

\textbf{\textsc{Engenheiro de Software Embarcado}} -- \textbf{Agres Sistemas Eletrônicos} \hfill \textit{06/2019 -- 01/2021}

{\small
\begin{itemize}[leftmargin=*,label=\large\textbullet]
    \setlength\itemsep{-0.2em}
    \item Contribuí para o desenvolvimento de uma aplicação embarcada para tratores agrícolas.
    \item Conduzi entrevistas e coletei requisitos do sistema.
    \item Desenvolvi uma aplicação em C++ utilizando o framework Qt 5.
    \item Aprimorei a arquitetura orientada a objetos aplicando os princípios SOLID.
    \item Implementei a interface do usuário com QML seguindo o padrão MVC.
    \item Colaborei em uma equipe Ágil, seguindo Scrum.
    \item Utilizei Git para controle de versão.
\end{itemize}
}

\textbf{\textsc{Engenheiro de Software}} -- \textbf{ExxonMobil BSC Brasil} \hfill \textit{11/2017 -- 06/2019}

{\small
\begin{itemize}[leftmargin=*,label=\large\textbullet]
    \setlength\itemsep{-0.2em}
    \item Contribuí para o desenvolvimento de programas ABAP para a plataforma SAP ERP.
    \item Desenvolvi e mantive relatórios SAP ABAP.
    \item Criei formulários editáveis usando as tecnologias SAP Smartforms e Formulários Interativos da Adobe.
    \item Desenvolvi pequenas aplicações web utilitárias com HTML, CSS e JavaScript.
    \item Criei scripts, de forma independente, que automatizaram processos manuais em formulários, economizando centenas de horas.
\end{itemize}
}

\preheadingspacing
\section{Formação Acadêmica}
\postheadingspacing
\begin{tabular}{rl}
%2017 & Graduação em \textsc{Engenharia de Computação} \\ &\normalsize\textbf{Universidade Tecnológica Federal do Paraná}, Curitiba-PR\\&\\
%2009 & Técnico em \textsc{Informática} \\&\normalsize\textbf{Escola Técnica Estadual de São Paulo}, São Paulo-SP
% \multirow{2}{*}{
% \shortstack[r]{
% Ongoing\\
% Expected by 10/2022
% }} & \textsc{Agile Software Development} specialization \\ &\normalsize\textbf{Universidade Federal do Paraná - UFPR}, Curitiba-PR\\&\\
%Nota: com esse \rule{2cm}{0pt} consegui dar uma largura minima pra coluna
\rule{3.2cm}{0pt} 2023 & Especialização em \textsc{Desenvolvimento Ágil de Software} \\ &\normalsize\textbf{Universidade Federal do Paraná (UFPR)}, Curitiba/PR\\&\\
2017 & Graduação em \textsc{Engenharia de Computação} \\ &\normalsize\textbf{Universidade Tecnológica Federal do Paraná (UTFPR)}, Curitiba/PR\\&\\
% 2009 & \textsc{Computer Technician} course \\ &\normalsize\textbf{Escola Técnica Estadual de São Paulo (ETESP)}, São Paulo-SP
\end{tabular}

%Section: Languages
%\section{L\'inguas}
%\begin{tabular}{rl}
%\textsc{Inglês:}&Fluente para leitura, escrita e conversação\\
%\textsc{Alemão:}&Básico para leitura, escrita e conversação\\
%\end{tabular}



\end{document}
